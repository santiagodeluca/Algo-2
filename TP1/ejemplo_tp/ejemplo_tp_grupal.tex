\documentclass[10pt,a4paper]{article}

\input{AEDmacros}
\usepackage{caratula} % Version modificada para usar las macros de algo1 de ~> https://github.com/bcardiff/dc-tex


\titulo{Descripci\'on del tp}
\subtitulo{Subtítulo del tp}

\fecha{\today}

\materia{Materia de la carrera}
\grupo{Grupo 42}

\integrante{Apellido, Nombre1}{001/01}{email1@dominio.com}
\integrante{Apellido, Nombre2}{002/01}{email2@dominio.com}
\integrante{Apellido, Nombre3}{003/01}{email3@dominio.com}
\integrante{Apellido, Nombre4}{004/01}{email4@dominio.com}
% Pongan cuantos integrantes quieran

% Declaramos donde van a estar las figuras
% No es obligatorio, pero suele ser comodo
\graphicspath{{../static/}}

\begin{document}

\maketitle

\section{Especifiación}
\subsection{Ejercicio 1}

\begin{proc}{grandesCiudades}{\In ciudades : \TLista{Ciudad}}{\TLista{Ciudad}}
	\requiere{noHayNombresRepetidos(ciudades) \land habitantesNoNegativos(ciudades)}
	\asegura{C \in res \iff (C \in ciudades \land C.habitantes > 50.000)}
\end{proc}

\pred{noHayNombresRepetidos}{\In listaDeCiudades : \TLista{Ciudad}}
{\paraTodo[unalinea]{elemento}{Ciudad}{elemento \in listaDeCiudades \implicaLuego \\
(\paraTodo[unalinea]{otroElmento}{Ciudad}{(otroElemento \in listaDeCiudades \land otroElemento \neq elemento) \implicaLuego \\ (elemento.nombre \neq otroElemento.nombre)})}}

\pred{habitantesNoNegativos}{\In listaDeCiudades : \TLista{Ciudad}}
{\paraTodo[unalinea]{elemento}{Ciudad}{elemento \in listaDeCiudades \implicaLuego elemento.habitantes \ge 0}}

\subsection{Ejercicio 2}

\begin{proc}{sumaDeHabitantes}{\In menoresDeCiudades : \TLista{Ciudad}, \In mayoresDeCiudades : \TLista{Ciudad}}{\TLista{Ciudad}}
	\requiere{noHayNombresRepetidos(mayoresDeCiudades) \land habitantesNoNegativos(mayoresDeCiudades) \land \\
	noHayNombresRepetidos(menoresDeCiudades) \land habitantesNoNegativos(menoresDeCiudades)}
	\requiere{|menoresDeCiudades| = |mayoresDeCiudades| \yLuego \\ mismosNombres(menoresDeCiudades, mayoresDeCiudades)}
	\asegura{|res| = |menoresDeCiudades| \land mismosNombres(res, menoresDeCiudades) \land \\ habitantesSumados(res, menoresDeCiudades, mayoresDeCiudades)}
\end{proc}

\pred{mismosNombres}{\In listaDeCiudades1 : \TLista{Ciudad}, \In listaDeCiudades2 : \TLista{Ciudad}}
{\paraTodo[unalinea]{elemento}{Ciudad}{elemento \in listaDeCiudades1 \implicaLuego \\
 \existe[unalinea]{otroElemento}{Ciudad}{otroElemento \in listaDeCiudades2 \yLuego elemento.nombre = otroElemento.nombre }}}

\pred{habitantesSumados}{\In listaSuma : \TLista{Ciudad}, \In listaMenores : \TLista{Ciudad}, \In listaMayores : \TLista{Ciudad}}
{\paraTodo[unalinea]{ciudadSuma}{Ciudad}{ciudadSuma \in listaSuma \implicaLuego \\
 \existe[unalinea]{ciudadMenor, ciudadMayor}{Ciudad}{(ciudadMenor \in listaMenores \land ciudadMayor \in listaMayores) \land \\ 
 ciudadSuma.nombre = ciudadMenor.nomobre \land ciudadSuma.nombre = ciudadMenor.nombre \land \\
 ciudadSuma.habitantes = ciudadMenor.habitantes + ciudadMayor.habitantes}}}

\subsection{Ejercicio 3}

\begin{proc}{hayCamino}{\In distancias : \TLista{\TLista{\ent}}, \In desde : \ent, \In hasta : \ent}{\bool}
	\requiere{esCuadrada(distancias) \yLuego \\ (esSimetrica(distancias) \land \\ diagonalEsCero(distancias) \land \\ valoresNoNegativos(distancias))}
	\asegura{res = True \iff \existe[unalinea]{camino}{\TLista{\ent}}{|camino| \geq 2 \wedge (esCamino(distancias, desde, hasta, camino))}}
\end{proc}

\pred{esCuadrada}{\In matriz : \TLista{\TLista{\ent}}}
{\paraTodo[unalinea]{fila}{\ent}{0 \leq fila < |matriz| \implicaLuego |matriz[fila]| = |matriz|}}

\pred{esSimetrica}{\In matriz : \TLista{\TLista{\ent}}}
{\paraTodo[unalinea]{i, j}{\ent}{0 \leq i, j < |matriz| \implicaLuego matriz[i][j] = matriz[j][i]}}

\pred{diagonalEsCero}{\In matriz : \TLista{\TLista{\ent}}}
{\paraTodo[unalinea]{i}{\ent}{0 \leq i < |matriz| \implicaLuego matriz[i][i] = 0}}

\pred{valoresNoNegativos}{\In matriz : \TLista{\TLista{\ent}}}
{\paraTodo[unalinea]{i, j}{\ent}{0 \leq i, j < |matriz| \implicaLuego matriz[i][j] \ge 0}}

\pred{esCamino}{\In distancias : \TLista{\TLista{\ent}}, \In desde : \ent, \In hasta : \ent, \In camino : \TLista{\ent}}
{ciudadesValidas(camino, |distancia|) \yLuego \\
(paresConectados(distancias, camino) \land \\
camino[0] = desde \land\\
camino[|camino| - 1] = hasta)} 

\pred{paresConectados}{\In distancias : \TLista{\TLista{\ent}}, \In camino : \TLista{\ent}}
{\paraTodo[unalinea]{i}{\ent}{0 \leq i < |camino| - 1 \implicaLuego distancias[camino[i]][camino[i + 1]] > 0}}

\pred{ciudadesValidas}{\In camino : \TLista{\ent}, \In longitud : \ent}
{\paraTodo[unalinea]{i}{\ent}{0 \leq i < |camino| \implicaLuego camino[i] < longitud}}


\subsection{Ejercicio 4}

Para este ejercicio usaremos que para cualquier matriz $M$ que cumpla que $M = PDP^{-1}$ con P una matriz y 
D una matriz diagonal, es decir, que todos sus valores son cero menos en la diagonal, se cumple que la diagonal
de D tiene los autovalores de la matriz $M$ y por lo tanto $M^{n} = PD^{n}P^{-1}$. \\
Notar que no todas las matrices son diagonalizables, pero sí lo son aquellas que son cuadradas y simétricas,
es decir, que verifican $M = M^{T}$. Todas las matrices que se introducen a este programa lo son.

\begin{proc}{cantidadCaminosNSaltos}{\Inout conexion : \TLista{\TLista{\ent}}, \In n : \ent}{}
	\requiere{conexion = C_{0}}
	\requiere{esCuadrada(conexion) \yLuego \\ (esSimetrica(conexion) \land \\ diagonalEsCero(conexion) \land \\ todosCerosOUnos(conexion))}
	\asegura{\existe[]{P, PInversa, diagonal, diagonalElevada}{\TLista{\TLista{\ent}}}{(|C_{0}| = |P| \land
	|C_{0}| = |PInversa| \land |C_{0}| = |diagonal| \land |C_{0}| = |diagonalElevada| \land \\
	esCuadrada(P) \land esCuadrada(PInversa) \land esCuadrada(diagonal) \land esCuadrada(diagonalElevada)) \yLuego \\
	esInversa(P, PInversa) \land \\
	esDiagonal(diagonal) \land \\
	esDiagonalElevada(diagonal, diagonalElevada, n) \land \\
	esProductoTripleDeMatrices(P, diagonal, PInversa, C_{0}) \land \\
	esProductoTripleDeMatrices(P, diagonalElevada, PInversa, conexion)}}
\end{proc}

\pred{todosCerosOUnos}{\In matriz : \TLista{\TLista{\ent}}}
{\paraTodo[unalinea]{i, j}{\ent}{0 \leq i, j < |matriz| \implicaLuego (matriz[i][j] = 0 \lor matriz[i][j] = 1)}}

Para el siguiente predicado usaremos que $A*A^{-1} = I$ donde I es la matriz identidad.

\pred{esInversa}{\In matrizOriginal : \TLista{\TLista{\ent}}, \In matrizInversa : \TLista{\TLista{\ent}}}
{\paraTodo[unalinea]{i, j}{\ent}{0 \leq i, j < |matrizOriginal| \implicaLuego \\ 
((i = j \land \sum\nolimits_{k=0}^{|matrizOriginal| - 1} (matrizOriginal[i][k] * matrizInversa[k][j]) = 1) \lor \\
(i \neq j \land \sum\nolimits_{k=0}^{|matrizOriginal| - 1} (matrizOriginal[i][k] * matrizInversa[k][j]) = 0))}}

\pred{esDiagonal}{\In matriz : \TLista{\TLista{\ent}}}
{\paraTodo[unalinea]{i, j}{\ent}{(0 \leq i, j < |matriz| \land i \neq j) \implicaLuego matriz[i][j] = 0}}

\pred{esDiagonalElevada}{\In matrizDiagonal : \TLista{\TLista{\ent}}, \In matrizDiagonalElevada : \TLista{\TLista{\ent}}, \In n : \ent}
{\paraTodo[unalinea]{i}{\ent}{0 \leq i < |matrizDiagonal| \implicaLuego matrizDiagonalElevada[i][i] = (matrizDiagonal[i][i])^{n}}}

\pred{esProductoDeMatrices}{\In matrizFactor1 : \TLista{\TLista{\ent}}, \In matrizFactor2 : \TLista{\TLista{\ent}}, \In matrizProducto : \TLista{\TLista{\ent}}}
{\paraTodo[unalinea]{i, j}{\ent}{0 \leq i, j < |matrizProducto| \implicaLuego \\ 
(matrizProducto[i][j] = \sum\nolimits_{k=0}^{|matrizProducto| - 1} (matrizFactor1[i][k] * matrizFactor2[k][j]))}}

\pred{esProductoTripleDeMatrices}{\In matrizFactor1 : \TLista{\TLista{\ent}}, \In matrizFactor2 : \TLista{\TLista{\ent}}, \In matrizFactor3 : \TLista{\TLista{\ent}},
\In matrizProducto : \TLista{\TLista{\ent}}}
{\existe[unalinea]{matrizAuxiliar}{\TLista{\TLista{\ent}}}{(esCuadrada(matrizAuxiliar) \land |matrizAuxiliar| = |matrizFactor1|) \yLuego \\
(esProductoDeMatrices(matrizFactor1, matrizFactor2, matrizAuxiliar) \land \\
esProductoDeMatrices(matrizAuxiliar, matrizFactor3, matrizProducto))}}

\subsection{Ejercicio 5}

\begin{proc}{caminoMinimo}{\In origen : \ent, \In destino : \ent, \In distancias : \TLista{\TLista{\ent}}}{\TLista{\ent}}
	\requiere{esCuadrada(distancias) \yLuego \\ (esSimetrica(distancias) \land \\ diagonalEsCero(distancias) \land \\ valoresNoNegativos(distancias))}
	\asegura{(esCamino(res, distancias, origen, destino \land \\
	\paraTodo[unalinea]{camino}{\TLista{\ent}}{distanciaRecorrida (camino) \ge distanciaRecorrida (res)}))}
\end{proc}



Lo principal: las fórmulas. Se puede poner en una linea, como $x_i = x_{i-1} + x_{i-2}$, o ponerse más grande:

\begin{equation}
	\sum\limits_{i=0}^{n} i
	\label{eq:1}
\end{equation}

Y se pueden citar ecuaciones con \verb|\eqref{nombreDeEq}|: \eqref{eq:1}

Ejemplo de itemizado:

\begin{itemize}
	\item Item 1
	\item Item 2
	\item Item 3
\end{itemize}

Ejemplo de enumerado con menor distancia entre items:

\begin{enumerate} \setlength\itemsep{0cm}
	\item Item 1
	\item Item 2
	\item Item 3
\end{enumerate}

Podemos escribir mucho texto. Mucho texto. Mucho texto. Mucho texto. Mucho texto. Mucho texto. Mucho texto. Mucho texto. Mucho texto. Mucho texto. Mucho texto.

Otro párrafo. Otro párrafo. Otro párrafo. Otro párrafo. Otro párrafo. Otro párrafo. Otro párrafo. Otro párrafo. Otro párrafo. Otro párrafo. Otro párrafo. Otro párrafo. Otro párrafo.

\vspace{0.3cm}

Le agregamos una separación entre párrafos. Le agregamos una separación entre párrafos. Le agregamos una separación entre párrafos. Le agregamos una separación entre párrafos. Le agregamos una separación entre párrafos.

\vspace{0.3cm}

La tabla \ref{tab:ejemplo} es un ejemplo de cómo se hace una tabla.

\begin{table}[h!]
	\centering
	\begin{tabular}{||l c c r||} 
		\hline
		Col1 & Col2 & Col2 & Col3 \\ [0.5ex] 
		\hline\hline
		1 & 6 & 87837 & 787 \\ 
		2 & 7 & 78 & 5415 \\
		3 & 545 & 778 & 7507 \\
		4 & 545 & 18744 & 7560 \\
		5 & 88 & 788 & 6344 \\
		\hline
	\end{tabular}
	\caption{Ejemplo de tabla}
	\label{tab:ejemplo}
\end{table}


La figura \ref{fig:subfigs} es un ejemplo de cómo se agrega una imagen.

\begin{figure}[ht]
	\centering
	\includegraphics[width=0.6\textwidth]{logo_dc.jpg}
	\caption{Ejemplo de figura}
	\label{fig:ejemplo}
\end{figure}

\begin{figure}[ht!]
	\begin{subfigure}{0.5\textwidth}
		\includegraphics[width=0.9\linewidth]{LaTeX-project} 
		\caption{Logo de LaTeX}
		\label{fig:subfig1}
	\end{subfigure}
	\begin{subfigure}{0.5\textwidth}
		\includegraphics[width=0.7\linewidth]{TeX}
		\caption{Logo de TeX}
		\label{fig:subfig2}
	\end{subfigure}
	\caption{Ejemplo para poner dos figuras juntas. Y citarlas por separado a (\subref{fig:subfig1}) y (\subref{fig:subfig2}).}
	% OJO: el caption siempre va antes del label
	\label{fig:subfigs}
\end{figure}



% Para hacer que quede todo en una misma linea, se puede usar minipage
%\begin{minipage}[t]{\textwidth}
	\begin{lstlisting}[caption={Ejemplo de código (usando los estilos de la cátedra, ver las macros para más detalles)},label=code:for]
res := 0;
i := 0;
while (i < s.size()) do
	res := res + s[i];
	i := i + 1
endwhile
	\end{lstlisting}
%\end{minipage}

Si se pone un label al \verb|lstlisting|, se puede referenciar: Código \ref{code:for}.


\subsection{Macros de la cátedra para especificar}

\begin{proc}{nombre}{\In paramIn : \nat, \Inout paramInout : \TLista{\ent}}{tipoRes}
	%    \modifica{parametro1, parametro2,..}
	\requiere{expresionBooleana1}
	\asegura{expresionBooleana2}
	\aux{auxiliar1}{parametros}{tipoRes}{expresion}
	\pred{pred1}{parametros}{expresion} 
\end{proc}

\aux{auxiliarSuelto}{parametros}{tipoRes} {\sum\limits_{i=0}^{n} i}

% \paraTodo{variable}{tipo}{expresion}
% \existe{variable}{tipo}{expresion}
% Pueden tener [unalinea] para que no se divida en varias lineas
\pred{predSuelto}{parametros}{\paraTodo[unalinea]{variable}{tipo}{algo \implicaLuego expresion}}
\pred{predSuelto}{parametros}{\existe[unalinea]{variable}{tipo}{(p \land q) \implicaLuego (r \yLuego s)}}



\end{document}
