\documentclass[10pt,a4paper]{article}
\usepackage{amssymb}

\usepackage{listings}
\usepackage{color}


\definecolor{dkgreen}{rgb}{0,0.6,0}
\definecolor{gray}{rgb}{0.5,0.5,0.5}
\definecolor{mauve}{rgb}{0.58,0,0.82}

\lstset{frame=tb,
  language=Java,
  mathescape,         
  literate={_0}{$_0$}{1}
  {_1}{$_1$}{1}
  {_2}{$_2$}{1},
  aboveskip=3mm,
  literate={'}{{'}}{1},
  belowskip=3mm,
  showstringspaces=false,
  columns=flexible,
  basicstyle={\small\ttfamily},
  numbers=none,
  numberstyle=\tiny\color{gray},
  keywordstyle=\color{blue},
  commentstyle=\color{dkgreen},
  stringstyle=\color{mauve},
  breaklines=true,
  breakatwhitespace=true,
  tabsize=3
}

\usepackage[spanish,activeacute,es-tabla]{babel}
\usepackage[utf8]{inputenc}
\usepackage{ifthen}
\usepackage{listings}
\usepackage{dsfont}
\usepackage{subcaption}
\usepackage{amsmath}
\usepackage[strict]{changepage}
\usepackage[top=1cm,bottom=2cm,left=1cm,right=1cm]{geometry}%
\usepackage{color}%
\newcommand{\tocarEspacios}{%
	\addtolength{\leftskip}{3em}%
	\setlength{\parindent}{0em}%
}

% Especificacion de procs

\newcommand{\In}{\textsf{in }}
\newcommand{\Out}{\textsf{out }}
\newcommand{\Inout}{\textsf{inout }}

\newcommand{\encabezadoDeProc}[4]{%
	% Ponemos la palabrita problema en tt
	%  \noindent%
	{\normalfont\bfseries\ttfamily proc}%
	% Ponemos el nombre del problema
	\ %
	{\normalfont\ttfamily #2}%
	\
	% Ponemos los parametros
	(#3)%
	\ifthenelse{\equal{#4}{}}{}{%
		% Por ultimo, va el tipo del resultado
		\ : #4}
}

\newenvironment{proc}[4][res]{%
	
	% El parametro 1 (opcional) es el nombre del resultado
	% El parametro 2 es el nombre del problema
	% El parametro 3 son los parametros
	% El parametro 4 es el tipo del resultado
	% Preambulo del ambiente problema
	% Tenemos que definir los comandos requiere, asegura, modifica y aux
	\newcommand{\requiere}[2][]{%
		{\normalfont\bfseries\ttfamily requiere}%
		\ifthenelse{\equal{##1}{}}{}{\ {\normalfont\ttfamily ##1} :}\ %
		\{\ensuremath{##2}\}%
		{\normalfont\bfseries\,\par}%
	}
	\newcommand{\asegura}[2][]{%
		{\normalfont\bfseries\ttfamily asegura}%
		\ifthenelse{\equal{##1}{}}{}{\ {\normalfont\ttfamily ##1} :}\
		\{\ensuremath{##2}\}%
		{\normalfont\bfseries\,\par}%
	}
	\renewcommand{\aux}[4]{%
		{\normalfont\bfseries\ttfamily aux\ }%
		{\normalfont\ttfamily ##1}%
		\ifthenelse{\equal{##2}{}}{}{\ (##2)}\ : ##3\, = \ensuremath{##4}%
		{\normalfont\bfseries\,;\par}%
	}
	\renewcommand{\pred}[3]{%
		{\normalfont\bfseries\ttfamily pred }%
		{\normalfont\ttfamily ##1}%
		\ifthenelse{\equal{##2}{}}{}{\ (##2) }%
		\{%
		\begin{adjustwidth}{+5em}{}
			\ensuremath{##3}
		\end{adjustwidth}
		\}%
		{\normalfont\bfseries\,\par}%
	}
	
	\newcommand{\res}{#1}
	\vspace{1ex}
	\noindent
	\encabezadoDeProc{#1}{#2}{#3}{#4}
	% Abrimos la llave
	\par%
	\tocarEspacios
}
{
	% Cerramos la llave
	\vspace{1ex}
}

\newcommand{\aux}[4]{%
	{\normalfont\bfseries\ttfamily\noindent aux\ }%
	{\normalfont\ttfamily #1}%
	\ifthenelse{\equal{#2}{}}{}{\ (#2)}\ : #3\, = \ensuremath{#4}%
	{\normalfont\bfseries\,;\par}%
}

\newcommand{\pred}[3]{%
	{\normalfont\bfseries\ttfamily\noindent pred }%
	{\normalfont\ttfamily #1}%
	\ifthenelse{\equal{#2}{}}{}{\ (#2) }%
	\{%
	\begin{adjustwidth}{+2em}{}
		\ensuremath{#3}
	\end{adjustwidth}
	\}%
	{\normalfont\bfseries\,\par}%
}

% Tipos

\newcommand{\nat}{\ensuremath{\mathds{N}}}
\newcommand{\ent}{\ensuremath{\mathds{Z}}}
\newcommand{\float}{\ensuremath{\mathds{R}}}
\newcommand{\bool}{\ensuremath{\mathsf{Bool}}}
\newcommand{\cha}{\ensuremath{\mathsf{Char}}}
\newcommand{\str}{\ensuremath{\mathsf{String}}}

% Logica

\newcommand{\True}{\ensuremath{\mathrm{true}}}
\newcommand{\False}{\ensuremath{\mathrm{false}}}
\newcommand{\Then}{\ensuremath{\rightarrow}}
\newcommand{\Iff}{\ensuremath{\leftrightarrow}}
\newcommand{\implica}{\ensuremath{\longrightarrow}}
\newcommand{\IfThenElse}[3]{\ensuremath{\mathsf{if}\ #1\ \mathsf{then}\ #2\ \mathsf{else}\ #3\ \mathsf{fi}}}
\newcommand{\yLuego}{\land _L}
\newcommand{\oLuego}{\lor _L}
\newcommand{\implicaLuego}{\implica _L}

\newcommand{\cuantificador}[5]{%
	\ensuremath{(#2 #3: #4)\ (%
		\ifthenelse{\equal{#1}{unalinea}}{
			#5
		}{
			$ % exiting math mode
			\begin{adjustwidth}{+2em}{}
				$#5$%
			\end{adjustwidth}%
			$ % entering math mode
		}
		)}
}

\newcommand{\existe}[4][]{%
	\cuantificador{#1}{\exists}{#2}{#3}{#4}
}
\newcommand{\paraTodo}[4][]{%
	\cuantificador{#1}{\forall}{#2}{#3}{#4}
}

%listas

\newcommand{\TLista}[1]{\ensuremath{seq \langle #1\rangle}}
\newcommand{\lvacia}{\ensuremath{[\ ]}}
\newcommand{\lv}{\ensuremath{[\ ]}}
\newcommand{\longitud}[1]{\ensuremath{|#1|}}
\newcommand{\cons}[1]{\ensuremath{\mathsf{addFirst}}(#1)}
\newcommand{\indice}[1]{\ensuremath{\mathsf{indice}}(#1)}
\newcommand{\conc}[1]{\ensuremath{\mathsf{concat}}(#1)}
\newcommand{\cab}[1]{\ensuremath{\mathsf{head}}(#1)}
\newcommand{\cola}[1]{\ensuremath{\mathsf{tail}}(#1)}
\newcommand{\sub}[1]{\ensuremath{\mathsf{subseq}}(#1)}
\newcommand{\en}[1]{\ensuremath{\mathsf{en}}(#1)}
\newcommand{\cuenta}[2]{\mathsf{cuenta}\ensuremath{(#1, #2)}}
\newcommand{\suma}[1]{\mathsf{suma}(#1)}
\newcommand{\twodots}{\ensuremath{\mathrm{..}}}
\newcommand{\masmas}{\ensuremath{++}}
\newcommand{\matriz}[1]{\TLista{\TLista{#1}}}
\newcommand{\seqchar}{\TLista{\cha}}

\renewcommand{\lstlistingname}{Código}
\lstset{% general command to set parameter(s)
	language=Java,
	morekeywords={endif, endwhile, skip},
	basewidth={0.47em,0.40em},
	columns=fixed, fontadjust, resetmargins, xrightmargin=5pt, xleftmargin=15pt,
	flexiblecolumns=false, tabsize=4, breaklines, breakatwhitespace=false, extendedchars=true,
	numbers=left, numberstyle=\tiny, stepnumber=1, numbersep=9pt,
	frame=l, framesep=3pt,
	captionpos=b,
}


\begin{document}
\title{AED 2 - Práctica 6}
\author{Santiago De Luca}

\maketitle

\section{Practica 6}
Aviso que no soy docente así que no hay nada que te asegura que los ejercicios estén bien ni mucho menos que sean la "solución esperada".
Por otro lado, hay varios ejercicios que tienen más de una interpretación y estructura a elegir posibles así que puede haber muchas soluciones correctas distintas :)
\subsection{Ejercicio 1}


\begin{lstlisting}
	proc agregarAtras(inout L: listaEnlazada<T>, in T:T) // O(1)
		Nodo := new NodoLista<T>
		Nodo.valor := T
		Nodo.siguiente := Null
		if L.longitud == 0 then
			L.primero := Nodo
			L.ultimo := Nodo
		else 
			L.ultimo.siguiente := Nodo
			L.ultimo = Nodo
		endif
		L.longitud ++

	proc obtener(in L:listaEnlazada, in i : Z) : T
		actual := L.primero
		j := 0
		while j < i // O(n) -> todo el proc esta en O(n)
			actual := actual.siguiente
			j++
		endwhile
		return actual.valor

	proc eliminar(inout : listaEnlazada<T>, in i : Z)
		if L.longitud := 1
			L.primero := null
			L.ultimo := null
		else
			actual := L.primero
			j := 0
			while j < i-1 // O(n) -> todo el proc esta en O(n)
				actual := actual.siguiente
				j++
			endwhile
			actual.siguiente := actual.siguiente.siguiente
			L.longitud := L.longitud - 1
		endif

	proc concatenar (inout L1:listaEnlazada<T>, in L2:listaEnlazada<T>) //O(1)
		L1.ultimo.siguiente := L2.primero
		L1.ulitmo := L2.ultimo
		L1.longitud := L1.longitud + L2.longitud
\end{lstlisting}
 
\pagebreak

\subsection{Ejercicio 2}

\begin{lstlisting}
	Modulo ConjuntoArr<T> implementa ConjuntoAcotado<T>{
		Var datos: Array<T>
		Var tam : int
		
\end{lstlisting}

\pred{invRep}{C : $ConjuntoArr<T>$}
{0 \le C.tam \le C.datos.length \\
\land noHayRepetidos(C)}

\pred{noHayRepetidos}{C : $ConjuntoArr<T>$}
{\paraTodo[unalinea]{i, j}{Z}{(0 \le i,j < C.tam \land i \neq j) \implicaLuego C.datos[i] \neq C.datos[j]}}

\pred{Abs}{C : $ConjuntoArr<T>$, C' : $ConjuntoAcotado<T>$}
{C.datos.length = C'.capacidad \land C.tam = |C'.elems|\\ 
\paraTodo[unalinea]{e}{T}{e \in C'.elems \iff pertenece(subseq(C.datos, 0, C.tam), e)}}

\pred{pertenece}{A : $Array<T>$, e : $T$}
{\existe[unalinea]{i}{Z}{0 \le i < A.length \implicaLuego A[i] = e}}

\begin{lstlisting}
	proc conjVacio(in capacidad:Z) : ConjuntoArr<T>
		res := new conjArr<T>
		res.datos := new Array<T>[capacidad]
		res.tam := 0
		return res

	proc pertenece(in C:ConjuntoArr<T>, in e:T) : bool
		res := False
		i := 0
		while i < C.tam
			if C.datos[i] == e then
				res := True
			endif
			i++
		endwhile
		return res

	proc agregar(inout C:ConjuntoArr<T>, in e:T) 
		if !pertenece(C,e) then
			C.datos[C.tam] := e
			C.tam++
		endif

	proc unir(inout C:ConjuntoArr<T>, in C':ConjuntoArr<T>)
		i := 0
		while i < C'.tam
			agregar(C,C'.datos[i])
			i++
		endwhile

	proc sacar(inout C:ConjuntoArr<T>, in e:T)
		if pertenece(C,e) then
			if C.tam $\le$ 1 then
				C.tam := 0
			else
				i := 0 
				while C.datos[i] != e
					i++
				endwhile
				C.datos[i] := C.datos[C.tam - 1]
				C.tam := C.tam - 1
			endif
		endif


	// este se ve raro por la forma en la que implemente sacar
	proc intersecar(inout C:ConjuntoArr<T>, in C':ConjuntoArr<T>) 
		i := 0 
		while i < C.tam
			if !pertenece(C', C.datos[i]) then
				sacar(C,C.datos[i])
			else
				i++
			endif
		endwhile
	}
\end{lstlisting}

\subsection{Ejercicio 3}

\begin{lstlisting}
	Modulo ConjuntoLista<T> implementa Conjunto<T>{
		Var datos: listaEnlazada<T>
		Var tam : int
		
\end{lstlisting}

Según el apunte, en el invRep y Abs cuando se predica sobre un módulo que es variable de control de otro módulo 
hay que hacer referencia a los observadores del TAD que implementa. listaEnlazada implementa Secuencia y el TAD secuencia 
está observado como una secuencia s.

\pred{invRep}{C : $ConjuntoLista<T>$}
{|C.datos.s| = C.tam \land noHayRepetidos(C.datos)}

\pred{noHayRepetidos}{L : $listaEnlazada<T>$}
{\paraTodo[unalinea]{i, j}{Z}{(0 \le i,j < |L.s| \land i \neq j) \implicaLuego L.s[i] \neq L.s[j]}}

\pred{Abs}{C : $ConjuntoLista<T>$, C' : $Conjunto<T>$}
{C.tam = |C'.elems| \land \\ 
\paraTodo[unalinea]{e}{T}{e \in C'.elems \iff e \in C.datos.s}}

\begin{lstlisting}
	proc conjVacio() : ConjuntoLista<T> // O(1)
		res := new ConjuntoLista<T>
		res.datos := listaVacia()
		res.tam := 0


	// asumo que iterador.siguiente no solo devuelve el elemento sino que mueve el puntero al siguiente (eso interpreto del apunte, no esta la funcion iterador.avanzar)
	proc pertenece(in C:ConjuntoLista<T>, in e:T) : bool // O(n)
		res := False
		it := itereador(C.datos) 
		while it.HaySiguiente
			if it.siguiente == e then 
				res := True
			endif
		endwhile
		return res

	proc agregar(inout C:ConjuntoLista<T>, in e:T) // O(n)
		if !pertenece(C,e) then
			C.datos.agregarAtras(e)
			C.tam++
		endif

	proc agregarRapido(inout C:ConjuntoLista<T>, in e:T) // O(1)
		C.datos.agregarAtras(e)
		C.tam++

	proc sacar(inout C:ConjuntoLista<T>, in e:T) // O(n)
		if pertenece(C,e) then
			i := 0 
			it := iterador(C.datos)
			while it.HaySiguiente && it.siguiente != e
				i++
			endwhile
			C.datos.eliminar(i)
			C.tam--
		endif

	proc unir(inout C:ConjuntoLista<T>, in C':ConjuntoLista<T>) // O(C.tam * C'.tam)
		it := iterador(C'.datos)
		while it.HaySiguiente // se ejecuta O(C'.tam) veces
			agregar(C,it.siguiente) // esta en O(C.tam)
		endwhile

	proc intersecar(inout C:ConjuntoLista<T>, in C':ConjuntoLista<T>) // O(C.tam * (C.tam + C'tam))
		it := iterador(C.datos)
		while it.HaySiguiente // O(C.tam)
			e := it.siguiente
			if !pertenece(C',e) then // O(C'.tam)
				sacar(C,e) // O(C.tam)
				C'.tam--
			endif
		endwhile

	}
\end{lstlisting}

\subsection{Ejercicio 4}
Este ejercicio me costó MAL. Así que es larguísimo. La consigna no era mega detallada, eso me confundió también.

\begin{lstlisting}
	Modulo Indice implementa Conjunto<Tupla<Z,Z,Z>>{
		Var datos: Array<Tupla<Z,Z,Z>>
		Var indice1: Array<Z>
		Var indice2: Array<Z>
\end{lstlisting}

\pred{invRep}{I : $Indice$}
{(I.datos.length = I.indice1.length \land I.datos.length = I.indice2.length) \yLuego \\
\paraTodo[unalinea]{k}{Z}{0 \le k < I.indice1.length \implica (k \in I.indice1 \land k \in I.indice2)} \land
componentesOrdenadas(I.datos, I.indice1, 1) \land componentesOrdenadas(I.datos, I.indice2, 2) \land \\
primeraComponenteOrdenada(I.datos)}

\pred{componentesOrdenadas}{datos : $Array<Tupla<Z,Z,Z>>$, indice :$Array<Z>$, componente : Z}
{\paraTodo[unalinea]{i, j}{Z}{0 \le i < j < indice.length \implicaLuego (datos[indice[i]]_{componente} < datos[indice[j]]_{componente})}}

\pred{primeraComponenteOrdenada}{datos : $Array<Tupla<Z,Z,Z>>$}
{\paraTodo[unalinea]{i, j}{Z}{0 \le i < j < datos.length \implicaLuego datos[i]_{0} < datos[j]_{0}}}

Estos últimos dos prcs implican que no hay valores repetidos en ninguna componente, gracias al menor estricto.
En particular, implica que no hay triplas repetidas(estoy implementando el TAD Conjunto).

\pred{Abs}{I : $Indice$, C : $Conjunto<Tupla<Z,Z,Z>>$}
{|C.elems| = I.datos.length \land \\ 
\paraTodo[unalinea]{e}{Tupla<Z,Z,Z>}{e \in C.elems \iff pertenece(C.datos, e)}}

\begin{lstlisting}
	proc indiceVacio() : Indice
		res := new Indice
		res.datos := new Array<Tupla<Z,Z,Z>>[0]
		res.indice1 := new Array<Z>[0]
		res.indice2 := new Array<Z>[0]
		return res


	// al principio interprete que habia que hacer era como un pertenece y aprovechar los indices para buscar, entonces lo hice con busqueda binaria. Como te imaginas, quedo larguisimo. La interpretacion mas corta es como si fuera un obtener, pero por indices: 
	proc buscarPor(in I:indice, in componente:Z, in posicion:Z) : Tupla<Z,Z,Z>
		if componente == 0 then
			return I.datos[posicion]
		else if componente == 1
			return I.datos[indice1[posicion]]
		else 
			return I.datos[indice2[posicion]]

	proc pertenece(in I:Indice, e:Tupla<Z,Z,Z>) : bool
		i := 0 
		while I.datos[i] != e 
			i++
		endwhile
		return i < I.datos.length

	proc agregar(inout I:Indice, in e:Tupla<Z,Z,Z>)
		if !pertenece(I,e) then
			nuevosDatos := new Array<Tupla<Z,Z,Z>[I.datos.length + 1]
			if I.datos.length == 0 then
				nuevosDatos[0] := e
				I.datos := nuevosDatos
				I.indice1 := new Array<Z>[1]
				I.indice2 := new Array<Z>[1]
				I.indice1[0] := 0
				I.indice2[0] := 0
			else 
				i := 0 
				while i < I.datos.length && I.datos[i]$_0$ < e$_0$
					nuevosDatos[i] := I.datos[i]
					i++
				endwhile
				pos := i 
				nuevosDatos[i] := e 
				i++
				while i < nuevosDatos.length
					nuevosDatos[i] := I.datos[i-1]
				endwhile
				j := 0
				while j < I.indice1.length
					if I.indice1[j] $\ge$ pos then
						I.indice1[j] := I.indice1[j] + 1
					endif
					if I.indice2[j] $\ge$ pos then
						I.indice2[j] := I.indice2[j] + 1
					endif
				endwhile

				I.datos := nuevosDatos

				I.indice1 := nuevoIndice(I.datos,I.indice1,pos,1)
				I.indice2 := nuevoIndice(I.datos,I.indice2,pos,2)
			endif
		endif
	
	proc nuevoIndice(in datos:Array<Tupla<Z,Z,Z>>, in indice:Array<T>, in pos:Z, in componente:Z) : Array<T>
		res := new Array<T>[indice.length + 1]
		i := 0
		while i < indice.length && datos[indice[i]]$_{componente}$ < datos[pos]$_{componente}$
			res[i] = indice[i]
			i++
		endwhile
		res[i] = pos
		i++
		while i < res.length
			res[i] := indice[i-1]
		endwhile
		return res

	proc sacar(inout I:Indice, in e:Tupla<Z,Z,Z>)
		if pertenece(I,e) && I.datos.length != 1 then
			posAEliminar := 0 
			while posAEliminar < I.datos.length && e != I.datos[posAEliminar]
				posAEliminar++
			endwhile
			nuevosDatos := new Array<Tupla<Z,Z,Z>>[I.datos.length - 1]
			i := 0 
			while i < posAEliminar
				nuevosDatos[i] := I.datos[i]
				i++
			endwhile
			i++
			while i < I.datos.length
				nuevosDatos[i-1] := I.datos[i]
				i++
			endwhile
			I.indice1 = eliminaIndice(I.indice1, posAEliminar)
			I.indice2 = eliminaIndice(I.indice2, posAEliminar)
			j := 0 
			while j < I.indice.1.length
				if I.indice1[j] > posAEliminar
					I.indice1[j] := I.indice1[j]-1
				endif
				if I.indice2[j] > posAEliminar
					I.indice2[j] := I.indice2[j]-1
				endif
			endwhile
			I.datos := nuevosDatos
		else if pertenece(I,e) && I.datos.length == 1 then
			I.datos := new Array<Tupla<Z,Z,Z>>[0]
			I.indice1 := new Array<Z>[0]
			I.indice2 := new Array<Z>[0]
		endif

	proc eliminaIndice(in indice:Array<Z>, in k:Z) : Array <Z>
		res := new Array<Z>[indice.length-1]
		i := 0
		while indice[i] != k
			res[i] := indice[i]
			i++
		endwhile
		i++
		while i < indice.length
			res[i-1] := indice[i]
			i++
		endwhile
		return res
	}

\end{lstlisting}

\subsection{Ejercicio 5}
Este ejercicio tiene varias formas de encarar el problema de diferenciar la cola vacía de la que tiene un elemento, yo hice una distinta
a la que vimos en clase, con una variable más:

\begin{lstlisting}
	Modulo BufferCircular<T> implementa ColaAcotada<T>{
		Var datos: Array<Z>
		Var vacia: bool
		Var inicio: Z
		Var fin: Z
\end{lstlisting}

\pred{invRep}{B : $BufferCircular<T>$}
{0 \le B.inicio < B.datos.length \land 0 \le B.fin < B.datos.length \land \\
(B.vacia = True \implica B.fin = B.inicio) \land \\
(B.inicio \neq B.fin \implica B.vacia = False)}

Asumo que el TAD ColaAcotada está observado con una secuencia s (igual que el TAD cola) y con un entero capacidad.
\pred{Abs}{B : $BufferCircular<T>$, C : $ColaAcotada<T>$}
{C.capacidad = B.datos.length \land \\ (C.s = <> \iff (B.vacia = True \land B.inicio = B.fin)) \land \\
((B.vacia = False \land B.inicio = B.fin) \implica |C.s| = 1) \land \\
(B.inicio < B.final \implicaLuego C.s = subseq(B.datos, B.inicio, B.final + 1)) \land \\
(B.final < B.inicio \implicaLuego C.s = subseq(B.datos, B.inicio, B.datos.length) ++ subseq(B.datos, 0, B.final + 1))}


\begin{lstlisting}
	proc nuevoBuffer(in cap:Z) : BufferCircular<T>
		res := new BufferCircular<T>
		res.datos := new Array<T>[cap]
		res.inicio := 0
		res.fin := 0
		res.vacio := True
		return res
	
	proc encolar(inout B:BufferCircular<T>, in e:T)
		if B.vacia then
			B.datos[B.fin] := e
			B.vacia := False
		else
			B.fin := (B.fin + 1) % B.datos.length
			B.datos[B.fin] := e
		endif

	
	proc desencolar(inout B:BufferCircular<T>) : T // B.vacia == False por requiere de desencolar
		res := B.datos[B.inicio]
		if B.inicio == B.fin then
			B.vacia := True
		else 
			B.inicio := (B.inicio + 1) % B.datos.length
		endif
		return res
	}
\end{lstlisting}

\subsection{Ejercicio 6}

\begin{lstlisting}
	Modulo PilaAr<T> implementa Pila<T>{
		Var datos: Array<T>
		Var tam: Z
		// agrego tam para que desapilar sea O(1), aunque apilar va a ser O(n) igual :(
\end{lstlisting}

\pred{invRep}{P : $PilaAr<T>$}
{0 \le P.tam \le P.datos.length}

El TAD Pila está observado con una secuencia s:

\pred{Abs}{P : $PilaAr<T>$, P' : $Pila<T>$}
{|P'.s| = P.tam \yLuego \\
\paraTodo[unalinea]{i}{Z}{0 \le i < P.tam \implicaLuego P.datos[i] = P's[i]}}

\begin{lstlisting}
	proc pilaVacia() : PilaAr<T> // O(1)
		res := new PilaArr<T>
		res.tam := 0 
		res.datos := new Array<T>[0]
		return res

	proc vacia(in P:PilaAr<T>) : bool // O(1)
		return P.tam == 0

	proc apilar(inout P:PilaAr<T>, in e:T) // O(n)
		if P.datos.length != P.tam then // esta rama es O(1), con esta estructura no siempre apilar es O(n) :)
			P.datos[P.tam] := e
			P.tam ++
		else
			nuevosDatos := new Array[P.tam + 1] // O(n)
			i := 0 
			while i < P.tam // O(n)
				nuevosDatos[i] := P.datos[i]
			endwhile
			nuevosDatos[P.tam] := e
			P.tam++
			P.datos := nuevosDatos
		endif
	proc desapilar(inout P:PilaAr<T>) : T // O(1)
		res := P.datos[P.tam - 1]
		P.tam := P.tam - 1
		return res
		
	proc tope(in P:PilaAr<T>) : T // O(1)
		res := P.datos[P.tam - 1]
		return res
	}
\end{lstlisting}

\begin{lstlisting}
	Modulo PilaEnlazada<T> implementa Pila<T>{
		Var datos: listaEnlazada<T>
		// en esta estructura se invierten las complejidades con respecto a la anterior, apilar es O(1) pero desapilar(y tope) O(n)
\end{lstlisting}

Pensé bastante y no encontré restricciones para invRep

\pred{invRep}{P : $PilaEnlazada<T>$}
{True}

listaEnlazada implementa el TAD secuencia que está observado con una secuencia s:

\pred{Abs}{P : $PilaEnlazada<T>$, P' : $Pila<T>$}
{|P'.s| = |P.datos.s| \land \\
\paraTodo[unalinea]{i}{Z}{0 \le i < |P.datos.s| \implicaLuego P.datos.s[i] = P's[i]}}

\begin{lstlisting}
	proc pilaVacia() : PilaEnlazada<T> // O(1)
		res := new PilaEnlazada<T>
		res.datos := nre listaEnlazada<T>
		return res

	proc vacia(in P:PilaEnlazada<T>, in e:T) : bool  // O(1)
		return vacia(P.datos)

	proc apilar(inout P:PilaEnlazada<T>, in e:T)  // O(1)
		P.datos.agregarAtras(e)

	proc desapilar(in P:PilaEnlazada<T>) : T // O(n)
		res := P.datos.obtener(P.datos.longitud - 1)
		P.datos.eliminar(P.datos.longitud - 1)
		return res

	proc tope(in P:PilaEnlazada<T>) : T // O(n)
		res := P.datos.obtener(P.datos.longitud - 1)
		return res
	}
\end{lstlisting}

\begin{lstlisting}
	Modulo DicArr<K,V> implementa Diccionario<K,V>{
		Var data: Array<Tupla<K,V>>
		Var tam: Z
\end{lstlisting}

\pred{invRep}{D : $DicArr<K,V>$}
{0 \le D.tam \le D.data.length \land \\
\paraTodo[unalinea]{i, j}{Z}{(0 \le i, j < D.data.length \land i \neq j) \implicaLuego D.data[i]_{0} \neq D.data[j]_{0}}}

El TAD Diccionario está observado con un diccionario dict.

\pred{Abs}{D : $DicArr<K,V>$, D' : $Diccionario<K,V>$}
{|D'.dict| = D.tam \land \\
\paraTodo[unalinea]{e}{Tupla<K,V>}{pertenece(D,e) \iff (e_{0} \in D'.dict \yLuego D'dict[e_{0}] = e_{1})}}

\pred{pertenece}{D : $DicArr<K,V>$, e: $Tupla<K,V>$}
{\existe[unalinea]{i}{Z}{0 \le i < D.tam \yLuego D.data[i] = e}}

\pagebreak

\begin{lstlisting}
	proc diccionarioVacio() : DicArr<K,V> // O(1)
		res := new DicArr<K,V>
		res.data := new Array<Tupla<K,V>>[0]
		res.tam := 0
		return res
	
	proc esta(in D:DicArr<K,V>, in k:K) : bool // O(n)
		i := 0
		while i < D.tam && D.data[i]$_0$ != k
			i++
		endwhile
		return i != D.tam

	// el proximo proc tiene como requiere que la clave este, o sea esta(D,k) == True
	proc dondeEsta(in D:DicArr<K,V>, in k:K) : Z // O(n)
		i := 0 
		while i < D.tam && D.data[i]$_0$ != k
			i++
		endwhile
		return i

	proc definir(inout D:DicArr<K,V>, in k:K, in v:V) // O(n)
		if esta(D,k) then // caso modificar clave existente
			D.data[dondeEsta(D,k)]$_1$ := v
		else if D.tam != D.data.length // caso agregar con espacio disponible
			D.data[D.tam]$_0$ := k
			D.data[D.tam]$_1$ := v
			D.tam++
		else // caso agregar sin espacio disponible 
			nuevosDatos := new Array<K,V>[D.tam+1]
			i := 0 
			while i < D.tam 
				nuevosDatos[i] := D.data[i]
				i++
			endwhile
			nuevosDatos[D.tam]$_0$ = k
			nuevosDatos[D.tam]$_1$ = v
			D.tam++
			D.data := nuevosDatos
		endif

	proc borrar(inout D:DicArr<K,V>, in k:K) // O(n)
		pos := dondeEsta(D,k)
		i := pos
		while i < D.tam - 1
			D.data[i] := D.data[i + 1]
			i++
		endwhile
		D.tam--
		
	proc obtener(in D:DicArr<K,V>, in k:K) : V
		return D.data[dondeEsta(D,k)]$_1$

	proc definirRapido(inout D: DicArr<K,V>, in k:K, in v:V) // O(n) :(
		if D.tam != D.data.length then 
			D.data[D.tam]$_0$ := k
			D.data[D.tam]$_1$ := v
			D.tam++
		else // si no tengo espacio disponible no me sirve de nada saber que el elemento no pertence (en esta estructura)
			nuevosDatos := new Array<K,V>[D.tam+1]
			i := 0 
			while i < D.tam 
				nuevosDatos[i] := D.data[i]
				i++
			endwhile
			nuevosDatos[D.tam]$_0$ = k
			nuevosDatos[D.tam]$_1$ = v
			D.tam++
			D.data := nuevosDatos
		endif
	
	proc tamanio(in D:DicArr<K,V>) : Z // O(1)
		return D.tam
	}
\end{lstlisting}


\begin{lstlisting}
	Modulo DictEnlazado<K,V> implementa Diccionario<K,V>{
		Var data: listaEnlazada<Tupla<K,V>>
\end{lstlisting}

\pred{invRep}{D : $DictEnlazado<K,V>$}
{\paraTodo[unalinea]{i, j}{Z}{(0 \le i, j < |D.data.s| \land i \neq j) \implicaLuego D.data.s[i]_{0} \neq D.data.s[j]_{0}}}

\pred{Abs}{D : $DictEnlazado<K,V>$, D' : $Diccionario<K,V>$}
{|D.data.s| = |D'.dict| \land \\
\paraTodo[unalinea]{e}{Tupla<K,V>}{e \in D.data.s \iff (e_{0} \in D'.dict \yLuego D'dict[e_{0}] = e_{1})}}

\begin{lstlisting}
	proc diccionarioVacio() : DictEnlazado<K,V> // O(1)
		res : = new DictEnlazado<K,V>
		res.data := new listaEnlazada<K,V>[0]
		return res

	proc esta(in D:DictEnlazado<K,V>, in k:K) : bool // O(n)
		i := 0 
		it := iterador(D.data)
		while it.HaySiguiente && it.siguiente$_0$ != k
			i++
		endwhile
		return i < D.data.length

	// el proximo proc tiene como requiere que la clave este, o sea esta(D,k) == True
	proc dondeEsta(in D: DictEnlazado<K,V>, in k:K) // O(n)
		i := 0 
		it := iterador(D.data)
		while it.HaySiguiente && it.siguiente$_0$ != k
			i++
		endwhile
		return i

	proc definir(inout D:DictEnlazado<K,V>, in k:K, in v:V) // O(n)
		if esta(D, k) then
			D.data.modicarPosicion(dondeEsta(D,k), Tupla<k,v>)
		else
			D.data.agregarAtras(<K,V>)
		endif

	proc obetener(in D:DictEnlazado<K,V>, in k:K) : V // O(n)
		return D.data.obtener(dondeEsta(D,k))$_1$

	proc borrar(inout D:DictEnlazado<K,V>, in k:K) // O(n)
		D.data.eliminar(dondeEsta(D,K))

	proc definirRapido(inout D:DictEnlazado<K,V>, in k:K, in v:V) // O(1) (no O(n) como la otra implementacion)
		D.data.agregarAtras(Tupla<K,V>)

	proc tamanio(in D:DictEnlazado<K,V>) : Z // O(1)
		return D.data.length
	}
\end{lstlisting}

\end{document}
